%%
%% Author: Dario Chinelli
%% begin: 2022-10-16
%% end:    2022-12-24
%%


%%  FILE DA ESEGUIRE 


% Preamble
\documentclass[10pt,a4paper]{report}
\newcommand{\fontsmall}{\fontsize{5pt}{6pt}\selectfont}

% Packages
\usepackage[subpreambles=true]{standalone}
\usepackage{import}
\usepackage[left=2.8cm, right=2.8cm, top=2cm, bottom=2cm]{geometry}
\usepackage[section]{placeins}
\usepackage{amsmath}
\usepackage{physics}
\usepackage{amssymb}
\usepackage{dirtytalk}
\usepackage[makeroom]{cancel}
\usepackage{xcolor}

% Document
\begin{document}

\begin{titlepage}
	\topskip0pt
	%\vspace*{\fill}
	\centering
	\vspace*{50mm}
	\huge \textbf{\textsc{Introduction to Quantum Field Theory}} \\
	\vspace*{3mm}
	\Large \textit{Notes re-written from lessons' attendance, 2022} \\
	\Large \textit{Prof. Polosa} \\
	\vspace*{4mm}
	\Large \textsc{Corso di Laurea Magistrale in Fisica - Sapienza}
\end{titlepage}
\restoregeometry


\FloatBarrier
\chapter{Settembre e Ottobre}
Appunti delle lezioni del Prof. Polosa relative al mese di settembre e ottobre 2022.
\import{chp/}{IQFT_ottobre}

\FloatBarrier
\chapter{Novembre}
Appunti delle lezioni del Prof. Polosa relative al mese di novembre 2022.
\import{chp/}{IQFT_novembre}

\FloatBarrier
\chapter{Dicembre}
Appunti delle lezioni del Prof. Polosa relative al mese di dicembre 2022.
\import{chp/}{IQFT_dicembre}


\end{document}
