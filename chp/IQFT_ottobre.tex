%%
%% Author: Dario Chinelli
%% begin 2022-10-16
%% last mod 2022-12-24
%%


%%  NON ESEGUIRE QUESTO FILE !


% Preamble
\documentclass[class=article]{standalone}

% Packages
\usepackage[subpreambles=true]{standalone}
\usepackage{import}
\usepackage{graphicx}
\usepackage{amsmath}


% Document
\begin{document}

\section{The rod}

Given a 1-dimensional rod composed by N-particles, linked each others with a "spring", the hamiltonian density is

\begin{equation*}
\begin{split}
\mathcal{H} = \frac{1}{2} \sum_{n=1}^{N} \Big[  P_n^2 + \Omega^2 (q_n - q_{n+1})^2 + \Omega_0^2 q_n^2 \Big]
\end{split}
\end{equation*}

where the last term $\Omega_0^2 q_n^2$ is relative to the equilibrium position of the n-particle.
The \emph{periodic boundaries conditions} to $N \to \infty$ and $a \to 0$.

On the other side we can write the Newtonian equation as 

\begin{equation*}
\begin{split}
& H = \frac{1}{2} \int_{0}^{L} dx \Big[ p^2(x) + v^2 \Big( \frac{\partial q(x)}{\partial x} \Big) \Big] \\
% & \mbox{where} \\
&  p(x) = \dot q(x) \\
&  \ddot q(x) = v^2 \frac{\partial^2 q(x)}{\partial x^2}
\end{split}
\end{equation*}

the solution inside the boundaries is
\begin{equation*}
\ddot q_n = \Omega^2 \Big( q_{n+1} + q_{n-1} -2 q_n \Big)
\end{equation*}


\paragraph{Normal modes} or normal frequencies
\begin{equation*}
\begin{split}
& q_n = \sum_j e^{i j n} \frac{Q_j}{\sqrt{N}} \\
& q(x) =  \frac{1}{\sqrt{a} } \sum_n e^{ \frac{2 \pi l}{N a} (n a) } \frac{Q_j}{\sqrt{N}}  = \frac{1}{\sqrt{a} }  \sum_k e^{i k x} \frac{Q_k}{\sqrt{N}} \\
& \quad\quad k = \frac{2 \pi l}{L} \\
& \Rightarrow\quad q(x) =  \sum_k e^{i k x} \frac{Q_k}{\sqrt{N a}} =  \sum_k e^{i k x} \frac{Q_k}{\sqrt{L}}
\end{split}
\end{equation*}

Considering now the Newtonian equation, $p^2(x) = \dot q^2(x) $, 
$ \sum_{n=1}^N e^{i n (j-j')} = \delta_{j, j'}$ where $j = \frac{2\pi l}{N}$, 
we can move from the sum to the integral using the following relation 
$ \sum_{n=1}^N \to \frac{1}{a} \int_{0}^{L} dx$ and this leads to
$\int_{0}^{L} dx e^{i (k-k') x} = L \delta_{k, k'}$ .

Somehow we may land on this following expression:
\begin{equation*}
\frac{1}{L} \sum_{k, k'} L \, \delta_{k, k'} Q_k \dot Q_{k'} = \sum_k Q_k \dot Q_{k} =   \sum_k | \dot Q_{k} |^2
\end{equation*}
To finally get a \emph{total classical description}: a discrete sum on a numerable set, as follow

\begin{equation*}
H = \frac{1}{2}  \sum_k  | \dot Q_{k} |^2 + k^2 v^2  | Q_{k} |^2
\end{equation*}

As before, notice that the sum $ \sum_{n=1}^N $ for $L \to \infty$ became $\frac{L}{2 \pi} \int dk$ and it admits waves.
Extending this to 3-dimensional space, it became
\begin{equation*}
 \sum_{\vec k} (\ldots) \quad (\mbox{when }L \to \infty) \quad \frac{V}{(2\pi)^3} \int d^3 k 
\end{equation*}

\paragraph{Quantum system:}  let's consider now a quantum system, a quantum description. \\
\emph{Postulate} the followings:

\begin{equation*}
\begin{split}
& \Big[ q_l , p_n \Big] = i \, \delta_{l\,n} \\
& \Big[ q_l , q_n \Big] = 0 \\
& \Big[ p_l , p_n \Big] = 0 \\
\end{split}\quad\quad\quad
\begin{split}
& \Big[ Q_l , P_n \Big] = i \, \delta_{ln} \\
& \Big[ Q_l , Q_n \Big] = 0 \\
& \Big[ P_l , P_n \Big] = 0
\end{split}\quad\quad\quad
\begin{split}
& \mbox{Where natural units are applied:} \\
& \quad\quad h = 1 \\
& \quad\quad c = 1
\end{split}
\end{equation*}

\begin{equation*}
\begin{split}
& \Rightarrow\quad\quad q_n^{\dagger} = q_n  \quad , \quad Q_{-j} = Q_{j}^{\dagger} \quad , \quad  P_{-j} = P_{j}^{\dagger} \\
\mbox{e.g.} & \quad q_n^{\dagger} = \Big( \sum_n e^{i n j} \frac{Q_j}{\sqrt{N}} \Big)^{\dagger} =  \sum_j e^{ - i n j} \frac{Q_j^{\dagger}}{\sqrt{N}} = q_n
\end{split}
\end{equation*}

From the hamiltonian
\begin{equation*}
\mathcal{H} = \frac{1}{2} \sum_j \Big[ P_j\,P_j^{\dagger} + \omega_j^2 \, Q_j\,Q_j^{\dagger}  \Big]
\end{equation*}
and given the following operators, we find $Q_j$ and $P_j$:
\begin{equation*}
\begin{split}
& a_j = \frac{1}{\sqrt{2\omega_j}} \Big( \omega_j Q_j + i\,P_j^{\dagger} \Big) \\
& a_j = \frac{1}{\sqrt{2\omega_j}} \Big( \omega_j Q_j^{\dagger} - i\,P_j \Big) 
\end{split}\quad\quad\quad\Rightarrow\quad\quad
\begin{split}
& Q_j = \frac{1}{\sqrt{2\omega_j}} \Big( a_j + a_{-j}^{\dagger} \Big) \\
& P_j = -i \Big( \frac{\omega_j}{2} \Big)^{\frac{1}{2}} \Big( a_{-j} - a_j^{\dagger} \Big)
\end{split}\quad\quad\quad\quad \mbox{and} \quad\quad
\begin{split}
& \mbox{keep in mind} \\
& \Big[ a_j , a_{j'} \Big] = \delta_{j \, j'}
\end{split}
\end{equation*}
\begin{equation*}
\begin{split}
& Q_j Q_j^{\dagger} = \frac{1}{2 \omega_j} \Big( a_j a_j^{\dagger} + a_j a_{-j} + a_{-j}^{\dagger} a_j^{\dagger} + a_{-j}^{\dagger} a_{-j} \Big) \\
& P_j P_j^{\dagger} =  \Big( \frac{\omega_j}{2} \Big)^{\frac{1}{4}} \Big( a_{-j} a_{-j}^{\dagger} - a_{-j} a_j - a_j^{\dagger}a_{-j}^{\dagger} + a_j^{\dagger} a_j \Big)
\end{split}
\end{equation*}
With these lasts results we may write the $\mathcal{H}$ as
\begin{equation*}
\begin{split}
\mathcal{H} & = \frac{1}{2} \sum_j \Big[ P_j\,P_j^{\dagger} + \omega_j^2 \, Q_j\,Q_j^{\dagger}  \Big]
 =  \frac{1}{2} \sum_j \omega_j \Big( a_j a_j^{\dagger} + a_j^{\dagger} a_j \Big) \\
& =  \frac{1}{2} \sum_j  \omega_j \Big( 2 \, a_j^{\dagger} a_j + 1 \Big)
= \sum_j  \omega_j \Big( a_j^{\dagger} a_j + \frac{1}{2} \Big)
\end{split}
\end{equation*}

\paragraph{Phonons description} Phonons are bosons, they're used to describe the quantum problem of the rod. Phonons are like photons but in the world of sound instead of light.
A n-particles system is defined with 
\begin{equation*}
\ket{n_1, n_2, n_3, \ldots } = (a_1^{\dagger})^{n_1}  (a_2^{\dagger})^{n_2}  (a_3^{\dagger})^{n_3} \ldots \ket{0}
\end{equation*}
and for the 1-d oscillator, with energy $E_n$, is as follows
\begin{equation*}
\begin{split}
& \ket{n} =  (a^{\dagger})^{n} \ket{0} \\
& E_n = \hbar \omega (n + \frac{1}{2}) \stackrel{nu}{=} \omega (n + \frac{1}{2}) 
\end{split}
\end{equation*}
For the phonons is easy to \emph{understand} which is the medium that make the transmission possible, but what about light?
For the light, photons, the medium may also be the \emph{vacuum}.

\begin{center}
Filosofeggiamo un po' ora:
\fbox{\begin{minipage}{42em}
\begin{center}
\large \emph{Particles are the excitation of the field} \\
\normalsize If you don't touch the piano it stays quiet, but if you play it it makes music ... song's particles.
\large \emph{The field is permanent.} \\
\normalsize \emph{Particles are not fixed, they live and die.} \\
You cannot touch or see the field that you're studying, but you can see/detect the particle that pop out from the field. \\
\large Fields are NOT real but mathematical description of the world.
\end{center}
\end{minipage}}
\end{center}

When you measure an energy it's always relative to an offset, a ground-state.
Because you want the \emph{vacuum} to be Lorentz invariant.
\begin{equation*}
\quad\Rightarrow\quad 
\Big( \mathcal{H} - E_0 \Big) \ket{0} = 0
\end{equation*}













\end{document}
